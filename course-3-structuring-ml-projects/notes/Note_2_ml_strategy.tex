\documentclass[12 pt]{article}

% Basic
\usepackage[utf8]{inputenc}
% \usepackage[spanish,mexico]{babel}

% Don't indent paragraphs, leave some space between them
\usepackage{parskip}
\usepackage{enumitem}
\usepackage{graphicx}
\usepackage{subfig}

% Math stuff
\usepackage{amsmath, amsfonts, mathtools, amsthm, amssymb}
\usepackage{breqn}
\usepackage{cancel}
\usepackage{etoolbox}
\usepackage{float}


% For neural networks
\usepackage{tikz}
\usetikzlibrary{matrix,chains,positioning,decorations.pathreplacing,arrows}

% Headers
\usepackage{fancyhdr}
\usepackage{lastpage}
\pagestyle{fancy}
\setlength{\headheight}{40pt}

% Commands
\newcommand\N{\ensuremath{\mathbb{N}}}
\newcommand\R{\ensuremath{\mathbb{R}}}
\newcommand\Z{\ensuremath{\mathbb{Z}}}
\newcommand\Q{\ensuremath{\mathbb{Q}}}
\newcommand\C{\ensuremath{\mathbb{C}}}

%Images
\usepackage{import}
\usepackage{xifthen}
\usepackage{pdfpages}
\usepackage{transparent}

\newcommand{\incfig}[1]{%
    \def\svgwidth{\columnwidth}
    \scalebox{.75}{\import{./figures/}{#1.pdf_tex}}
}

\newtheorem{teo}{Teorema}
\newtheorem{lema}{Lemma}

\newenvironment{solution}
  {\renewcommand\qedsymbol{$\blacksquare$}
  \begin{proof}[Proof]}
  {\end{proof}}
\renewcommand\qedsymbol{$\blacksquare$}


\begin{document}

\lhead{Hyperparameter tuning, Regularization \\ and Optimization}
\rhead{ Deep Learning specialization}
\cfoot{\thepage \ of \pageref{LastPage}}

\section*{Error analysis}

Error analysis:
\begin{itemize}
    \item Get ~100 mislabeled dev set examples
    \item Count up how many of them are of one particular class
\end{itemize}
Knowing that you can take decisions on which class mislabelling you should tackle. This
process gives you an estimate of how worthwhile it might be to work on each of 
the different categories of errors and it five syou a sense of the best options to pursue
(maybe blurry images, images of one particular category being mislabeled as another one, etc)

\textbf{Incorrectly labeled data:} It turns out that deep learning algorithms 
are quite robust to random errors in the training set; if the errors are reasonably 
random, then it's probably okay. Nevertheless, deep learning algorithms are less 
robust to systematic errors.

\textbf{Correcting incorrect dev/test sets:} It might only be worthwhile doing it after
evaluating the impact using error analysis. If you decide to do it some guidelines are:

\begin{itemize}
    \item Apply same process to dev and test sets to make sure they continue to come 
    from the same distribution. 
    \item Consider examining examples the algorithm got right as well as the ones it got 
    wrong.
    \item Train and dev/test data may now come from slightly different distributions. (this 
    might be okay)
\end{itemize}

\textbf{Build the first system quickly and then iterate:} Set up dev/test set and metrics.
Build an initial system quickly and then use bias/variance analysis and error analysis
to prioritize next steps.

\section*{Mismatched training and dev/test sets}

You might want to (purposely) use train and dev/test sets with different distributions since
sometimes you can't gather enough data to train your system and you might get data from different
sources to make a richer training. 

\textbf{Bias and Variance with mismatched data distributions} When training and testing 
data comes from different distributions you can no longer draw the conclusion that the 
algorithm is not generalizing if your dev error ir much bigger than the train error. 

For this problem is important to create a \textit{training-dev} set that has de same distribution
as the training (but it's not used for training). Looking at this error you can see if you 
have a generalization problem (your training-dev error is similar to the dev error) or a
distribution problem (you training-dev is small so the problem is because of the data mismatching)

\textbf{Addressing data mismatch}  Carry out manual error analysis and try to make the training
data more similar or so that it resambles the kind of data that produced the mistakes.

\section*{Transfer learning}

One of the most powerful ideas in deep learning is that sometimes you can take knowledge 
the neural network has learned from one task and apply that knowledge to a separate task.
This is knonw as transfer learning.

The basic idea is to take a neural network that was trained for a particular task, remove
the last layer and train it for the new (similar) task. Usually this is called \textit{finetuning}. 

When does transfer learning makes sense? Transfer learning makes sense when you have a 
lot of data for the problem you're transferring from and usually relatively less 
data for the problem you're transferring to. In general:

\begin{itemize}
    \item Task A ans task B have the same input x.
    \item You have a lot mode data for task A than for task B. 
    \item Low level features from A could be helpfull for learning B.
\end{itemize}

\section*{Multi task learning}

So whereas in transfer learning, you have a sequential process where you learn from 
task A and then transfer that to task B. In multi-task learning, you start off 
simultaneously, trying to have one neural network do several things at the same time. 
And then each of these task helps hopefully all of the other task.

When does multi-task learning makes sense?

\begin{itemize}
    \item Training on a set of tasks that could benefit from havind shared lower level
    features. 
    \item Usually: Amount of data you have for each task is quite similar. 
    \item Can train a big neural network to do well on all tasks. 
\end{itemize}

\section*{End-to-end deep learning}

There have been some data processing systems, or learning systems that require multiple 
stages of processing. And what end-to-end deep learning does, is it can take all those 
multiple stages, and replace it usually with just a single neural network.

\textbf{Pros and cons of end-to-end deep learning}
\begin{itemize}
    \item Pro: Let the data speak
    \item Pro: Less hand-designing of components needed.
    \item Con: May need large amounts of data. 
    \item Con: Excludes potentially useful hand-designed components. 
\end{itemize}

Key question: Do I have sufficient data to learn a function od the complexity needed
to map x to y?

\end{document}