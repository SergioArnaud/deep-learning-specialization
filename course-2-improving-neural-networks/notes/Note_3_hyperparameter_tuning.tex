\documentclass[12 pt]{article}

% Basic
\usepackage[utf8]{inputenc}
% \usepackage[spanish,mexico]{babel}

% Don't indent paragraphs, leave some space between them
\usepackage{parskip}
\usepackage{enumitem}

% Math stuff
\usepackage{amsmath, amsfonts, mathtools, amsthm, amssymb}
\usepackage{breqn}
\usepackage{cancel}
\usepackage{etoolbox}
\usepackage{float}


% For neural networks
\usepackage{tikz}
\usetikzlibrary{matrix,chains,positioning,decorations.pathreplacing,arrows}

% Headers
\usepackage{fancyhdr}
\usepackage{lastpage}
\pagestyle{fancy}
\setlength{\headheight}{40pt}

% Commands
\newcommand\N{\ensuremath{\mathbb{N}}}
\newcommand\R{\ensuremath{\mathbb{R}}}
\newcommand\Z{\ensuremath{\mathbb{Z}}}
\newcommand\Q{\ensuremath{\mathbb{Q}}}
\newcommand\C{\ensuremath{\mathbb{C}}}

%Images
\usepackage{import}
\usepackage{xifthen}
\usepackage{pdfpages}
\usepackage{transparent}

\newcommand{\incfig}[1]{%
    \def\svgwidth{\columnwidth}
    \scalebox{.75}{\import{./figures/}{#1.pdf_tex}}
}

\newtheorem{teo}{Teorema}
\newtheorem{lema}{Lemma}

\newenvironment{solution}
  {\renewcommand\qedsymbol{$\blacksquare$}
  \begin{proof}[Proof]}
  {\end{proof}}
\renewcommand\qedsymbol{$\blacksquare$}


\begin{document}

\lhead{Hyperparameter tuning, Regularization \\ and Optimization}
\rhead{ Deep Learning specialization}
\cfoot{\thepage \ of \pageref{LastPage}}

\subsection*{Hyperparameter tuning process}

Usually the most important hyper parameter is $\alpha$ the learning rate. Second in 
importance are the number of hidden units, the mini-batch size and the $\beta$ momentum
term. Third in importance are the number of layers and the learning rate decay.

Which hyperparemeters to try? Try random values; don't use a grid search. Also it's 
common to use the "coarse to fine" sampling schema where you start by sampling randomly
in a region, then pick a good subregion and continue sampling in smaller and smaller regions.

\textbf{Use an appropiate scale to pick hyperparameters:} Don't necessarily sample 
the hyperparameters randomly. For example if you are searching hyperparameters for the
learning rate $\alpha \in [.001, 1]$ uniformly then $90$\% of the values you're 
trying lie between $.1$ and $1$, it would be more appropiate to use a log scale and sample
uniformly from using that scale so that you explore more than 10\% between $[.001,.1]$.

\subsection*{Batch Normalization}

Batch normalization makes the hyperparameter search problem much easier and the 
neural network trained much more robust. The choice of hyperparameters is a much bigger 
range  of hyperparameters that work well, and will also enable you to much more easily train 
even very deep networks.

For any hidden layer, can we normalize the value $a^{[l]}$ to train $W^{[l+1]}, b^{[l+]}$ 
faster? That's the main idea of batch normalization, in the practice $z^{[l]}$ is 
the one that's normalized. 

Given $z^{[l]} = (z^{(1)}, \dots, z^{(m)})$:
\begin{align*}
    \mu_i &= \frac{1}{m} \sum_{i} z^{(i)} \\
    \sigma_i^2 &= \frac{1}{m} \sum_{i} (z^{(i)} - \mu_i)^2 \\
    z^{(i)}_{\text{norm}} &= \frac{z^{(i)} - \mu_i}{\sqrt{\sigma_i^2 + \epsilon}} \\
    \hat{z}^{(i)} &= \gamma z^{(i)}_{\text{norm}} + \beta
\end{align*}
Where $\gamma$ and $\beta$ are learnable parameters of the model that can modify the 
distribution of $z^{(i)}$ so that it lies in the range of values you need.

Batch normalization reduces the problem of the input values (of each layer) changing,
it causes these values to become more stable, so that the later layers of the 
neural network has more firm ground to stand on.

\textbf{Batch norm as Regularization} Each mini-batch is scaled by the mean and variance
computed on just that mini-batch, this adds some noice the values $z^{[l]}$ because
it's just computed on one mini-batch. This noisy procedure is similar to dropout 
having a slight regularization effect. 

\textbf{Batch norm at test time} During test time the estimations of $\mu$ and $\sigma^2$
are obtained using exponentially weighted averages (across mini-batches)

\subsection*{Multi-class classification}

Think of the problem of classificating across multiple ($C$) classes, instead of just two as
we've seen so far. This problem is called \textit{multi-class classficiation}, for it
we'll start to introduce a generalization of the logistic regression called softmax 
regression, softmax regression generalizes logistic regression to $C$ classes. 

The softmax activativation function is given by:
\begin{align*}
    t &= e^{(z^{[l]})} \\
    a_i^{[l]} &= \dfrac{t_i}{\sum_{i} t_i}
\end{align*}
\textbf{Training a softmax classifier,} the loss function and cost functions are given by:
\begin{align*}
    L(\hat{y},y) &= - \sum_{j=1}^C y_j \log{\hat{y_j}} \\
    J(W,b) &= \frac{1}{m} \sum_{i=1}^m L(\hat{y},y)
\end{align*}




\end{document}